\documentclass[10pt, a4paper, final]{article}
\usepackage[a4paper,width=185mm,top=20mm,bottom=20mm]{geometry}
\usepackage{pythonhighlight}
\usepackage{url}
\usepackage{lipsum}
\usepackage{amsmath}
\usepackage{multicol}
\setlength{\columnsep}{1cm}
\usepackage{graphicx}
\graphicspath{{../results/}{../immagini/}}
\DeclareGraphicsExtensions{.png}
\usepackage{wrapfig}
\usepackage[font=small,labelfont=bf]{caption}
\usepackage{setspace}
\usepackage{caption}
\usepackage{subcaption}
\usepackage{wrapfig}
\usepackage{float}
\usepackage{xcolor}
\usepackage[tikz]{mdframed}
\usepackage{colortbl}


\usepackage[T1]{fontenc}
\usepackage{ascii}
\usepackage{listings}

\definecolor{forestgreen}{rgb}{0.0, 0.27, 0.13}
%%
%% Julia definition (c) 2014 Jubobs
%%
\lstdefinelanguage{Julia}%
  {morekeywords={abstract,break,case,catch,const,continue,do,else,elseif,%
      end,export,false,for,function,immutable,import,importall,if,in,%
      macro,module,otherwise,quote,return,switch,true,try,type,typealias,%
      using,while},%
   sensitive=true,%
   alsoother={\$},%
   morecomment=[l]\#,%
   morecomment=[n]{\#=}{=\#},%
   morestring=[s]{"}{"},%
   morestring=[m]{'}{'},%
}[keywords,comments,strings]%

\lstset{%
    language         = Julia,
    basicstyle       = \ttfamily,
    keywordstyle     = \bfseries\color{blue},
    stringstyle      = \color{magenta},
    commentstyle     = \color{forestgreen},
    showstringspaces = false,
}


\definecolor{airforceblue}{rgb}{0.36, 0.54, 0.66}
\mdfsetup{
middlelinecolor=airforceblue,%
middlelinewidth=0.2pt,%
roundcorner=5pt}
\setlength{\abovedisplayskip}{3pt}
\setlength{\belowdisplayskip}{3pt}

\renewcommand{\baselinestretch}{1}


\title{\textsc{Report 1 - Arecchi's experiment analysis}}
\author{Francesco Lorenzi,      October 2020}
\date{}
\begin{document}
\maketitle
\vspace{-25pt}

\begin{center}
	\rule[0pt]{400pt}{0.5pt}
\end{center}
\vspace{-15pt}

\begin{multicols}{2}
\subsubsection*{Summary}
An interesting experiment developed by F. T. Arecchi, in 1965, show how the light type can be changed by means of a rough surface instead of a source interchange. The change from coherent to thermal light is verified using photon counting techniques. In this report a set of data, collected in a similar setup, is analyzed in order to verify the change of light type.

\subsection*{Experimental setup}
The experiment is carried out sending an He-Ne laser beam to a disk of sandpaper, selectively set to rotate by powering an electrical motor. The scattered beam is then measured by means of a SPAD, connected to a time-tagger that assign to each pulse, generated by the avalanche due to an incoming photon, the time of occurrence in a suitable machine time reference. 
\begin{mdframed}
    \begin{figure}[H]
    \centering
    \includegraphics[width = \textwidth]{../images/setup.png}
    \caption{Experimental setup scheme}
  \end{figure}
\end{mdframed}
 


The purpose of the experiment is to get a characterization of the \emph{type} of light in the two status of motion of the disk, by means of the photon arrival data provided by the time-tagger.
In fact the \emph{type} of light is a properity associated to the photon arrival statistic, which can be estimated from a list of arrival times.

In order to link radiometric quantities to photon counting, it is possible to use the following relations. If we divide the total acquisition time in time intervals, each of duration $T$, and a quasi monochromatic source of central frequency $\bar{\nu}$ with power of $P(\mathbf{r}, t)$ is impinging perpendicularly into an area element $ds$ of a total surface $\Sigma$, we expect a mean number of photons which is
\begin{equation*}
    \bar{n} = \int_{kT}^{(k+1)T} \int_\Sigma \frac{P(\mathbf{r}, t) ds}{h \bar{\nu} } dt
\end{equation*}
if the power is constant over the integration period $T$, we can define the (constant) mean photon flux as
\begin{equation*}
    \Phi = \int_\Sigma \frac{P(\mathbf{r}) ds}{h \bar{\nu} } 
\end{equation*}
so the mean photon count over $[kT, (k+1)T]$ will be $\bar{n} = \Phi T$.
This is all that can be said about the light if only it's (constant) intensity is known. If it is possible to count, as it is done, the impinging photons, some other properities become measurable.
The \emph{type} of light is called to be \textbf{coherent} if the number of photons which triggers an event in all time intervals $[kT, (k+1)T]$, is a random process of i.i.d. Poisson variables, each with probability distribution given by
\begin{equation*}
    p(n) = \frac{\bar{n}^n}{n!} e^{-\bar{n}}
\end{equation*}
so the interarrival time is a process of i.i.d. exponential variables.

To the opposite side, the \emph{type} of light is called to be \textbf{thermal} if the same count is described by a random process of i.i.d. Bose-Einstein variables, each with probability distribution given by
\begin{equation*}
    p(n) = \frac{1}{\bar{n} + 1} \left(\frac{\bar{n}}{\bar{n} + 1}\right)^n
\end{equation*}
we notice that the distribution resembles a Geometric distribution of parameter $1/(\bar{n}+1)$.

In specific we want to verify that in the in case of the \emph{still} disk the light is of \emph{coherent} type, and in the case of the \emph{spinning} disk it is of \emph{thermal} type.
\vspace{15pt}

% COHERENCE TIME
In the analysis of data we have a free parameter, namely $T$, the time interval of acquisition. The choice of $T$ is actually critical to the result of the experiment, for two main reasons:
\begin{enumerate}
    \item \emph{Insufficient sample:}
     a small $T$ will give place to a distribution of mostly empty intervals. Conversely, a too large $T$ will give poor repetition over each different counts value, which will be realized only few times. A suitable $T$ is chosen by trials and errors, in a qualitative way.
    \item \emph{Coherence time of sources:}
     even if the theory of coherence is out of the scope of this analysis, the measures of the count in each interval for the spinning disk shall be done within a given \emph{coherence time} $\tau_c$ of the scattered light, in which the field properities are supposed to stay constant. So another limitation is $T<\tau_c$.
\end{enumerate}

% % ENSEMBLE AVERAGE
% Some comments on the statistics are useful to understand better the problem: we may distinguish between \textsc{ensemble} average and \textsc{sample} average.

\subsection*{Analysis}
The objective of the experiment is to determine the type of light, regardless of the other light parameters. In fact, it is impractical to get a theoretical value of the mean photon flux across the detector, because of the scattering of the rough surface. 

Both Poisson and Bose-Einstein distributions are uniquely determined by their mean, so, in order to spot the distribution type of data, we estimate from the sample mean $\bar{n}$ the expected value of the random variables, so it is possible to reconstruct the theoretical distributions. 

At a later stage, the empirical moments are evaluated and compared to theoretical ones.
If we call $F(i)$ the empirical frequency of $i$ photon hits in the interval, an empirical probability density is found by normalizing by the total number of photons $N$: $p(i) := F(i) / N$ so the moments can be derived as follows
\begin{align*}
    \mu &= \sum_i n_i p(i) \\
    \sigma^2 &= \sum_i (n_i - \mu)^2 p(i) \\
    \tilde{\mu}_3 &= \frac{1}{\sigma^3}\sum_i (n_i - \mu)^3 p(i) \qquad \text{skewness}\\
    \tilde{\mu}_4 &= \frac{1}{\sigma^4}\sum_i (n_i - \mu)^4 p(i) \qquad \text{kurtosis}
\end{align*}
\emph{skewness} and \emph{kurtosis} are, respectively, the third and the fourth standardized moments.
The comparison between moments gives an interesting view on the similarity between empirical results and theoretical predictions.

The choice of the interval is set to $T = 15\mu s$.  
The results for the still disk are collected in Table \ref{still_mom}, whereas the results for the spinning disk are in Table \ref{spin_mom}. All the comparisons are completed with a relative error with respect to the theoretical value. 
\begin{mdframed}
    \begin{table}[H]
        \begin{tabular}{ |c||c|c|c| }
            \hline
             & Empirical & Poisson & Rel. error \\
            \hline
            \hline
            $\mu$            & 4.28 & - & - \\
            \hline
            $\sigma^2$       & 4.54 & 4.28 & 0.060 \\   
            \hline
            $\tilde{\mu}_3$  & 0.527& 0.483 & 0.090 \\ 
            \hline
            $\tilde{\mu}_4$  & 3.31 & 3.23 & 0.022 \\ 
            \hline
        \end{tabular}
        \caption{Still disk statistics}
        \label{still_mom}
    \end{table}
\end{mdframed}

\begin{mdframed}
    \begin{table}[H]
        \begin{tabular}{ |c||c|c|c| }
            \hline
             & Empirical & Bose-Einstein & Rel. error \\
            \hline
            \hline
            $\mu$            & 2.36 & - & - \\
            \hline
            $\sigma^2$       & 10.57 & 7.94 & 0.332 \\   
            \hline
            $\tilde{\mu}_3$  & 3.74 & 2.03 & 0.842 \\ 
            \hline
            $\tilde{\mu}_4$  & 28.15 & 9.12 & 2.08 \\ 
            \hline
        \end{tabular}
        \caption{Spinning disk statistics}
        \label{spin_mom}
    \end{table}
\end{mdframed}

The similarity between statistics is also proven by drawing the theoretical distributions along with the empirical probability histograms, as shown in the following figures. 

\end{multicols}
\vspace{50pt}
\begin{mdframed}
  \begin{figure}[H]
  \centering
  \includegraphics[width = 0.9\textwidth]{../images/still-15us.png}
  \caption{Empirical distribution with still disk, $T = 15\mu s$}
\end{figure}
\end{mdframed}

\begin{mdframed}
  \begin{figure}[H]
  \centering
  \includegraphics[width = 0.9\textwidth]{../images/spin-15us.png}
  \caption{Empirical distribution with spinning disk,  $T = 15\mu s$}
\end{figure}
\end{mdframed}


Furthermore, we observe the empirical distribution for a much longer $T$ ($T=200 \mu s$), for which case an interesting phenomena occurs: although the distribution for coherent light stays similar to a Poisson, the distribution for the thermal source is no more well approximated by a Bose-Einstein. 
\vspace{20pt}
\begin{mdframed}
  \begin{figure}[H]
  \centering
  \includegraphics[width = 0.8\textwidth]{../images/still-200us.png}
  \caption{Empirical distribution with still disk,  $T = 200\mu s$}
\end{figure}
\end{mdframed}
\vspace{20pt}
\begin{mdframed}
    \begin{figure}[H]
    \centering
    \includegraphics[width = 0.8\textwidth]{../images/spin-200us.png}
    \caption{Empirical distribution with still disk,  $T = 200\mu s$}
  \end{figure}
  \end{mdframed}

\begin{multicols}{2}
     
\subsection*{Interpretation of results}
As expected, at least for an adequate $T$, the distribution of photon counts is well distinct in the two cases of still and moving disk, and the statistical description, from the light type definition, is qualitatively confirmed by the data.

The odd result for the spinning disk with a long $T$ is due to the fact that such $T$ is above the coherence time for the spinning disk, so the count values are averaged, and the probability distribution no longer shows it's maximum at 0, but at an intermediate count. In order to fully comprehend this phenomena it is necessary to study further in the theory of coherence.

%A comment may be done about a common conception about thermal light is present in interpretation: a thermal light source is usually assimilated to a superposition of a large number of coherent sources, but it is also true that a sum of Poisson random variables is also Poissonian. So one may expect to have a Poissonian distribution also in the case of thermal light, but the experiment deny it. In fact the pitfall is that the thermal source can be assimilated to a large number of coherent sources of \emph{varying intensity} over time. In that case, it is no longer true that each source is represented well by a Poisson process, because of the variation of the intensity.

\subsection*{Conclusions}
Even if the theory of this phenomenon is not developed in depth in this report, the purpose of the analysis is met, and the main points of Arecchi experiment are shown. The work in photon statistics is a launchpad for the understanding of a more comprehensive theory for quantum optics.

\subsubsection*{Useful references}
A complete theory of the photoelectron statistics requires the machinery of quantum optics to some level. Also the theory of optical coherence plays a major role, and gives deep insights. 
The experiment on which this analysis is based is described on \textsc{F.T.Arecchi}, \emph{Measurement of the statistical distribution of Gaussian and Laser sources} (1965, Phys. Rev. Lett. vol.14 n.24, pages 912-916). 
A general exposition of coherence is fond in \textsc{Born, Wolf}, \emph{Principles of Optics} (2019, Cambridge University Press), whereas in \textsc{Mandel, Wolf}, \emph{Optical coherence and quantum optics} (2008, Cambridge University Press), in chapter 9, there is a semiclassical derivation of the photoelectron statistics.
Also, a introductory classical theory of coherence is described in \textsc{Someda}, \emph{Electromagnetic Waves} (2017, CRC Press).

\end{multicols}

\hrulefill
\subsection*{Code}
In order to carry out the analysis a relatively recent programming language is used. The Julia programming language provide a fast computing framework, and it's high level syntax make the shift from Matlab or Python simple. For further information consult: \url{https://julialang.org/}.

\renewcommand{\baselinestretch}{0.5}
\begin{mdframed}
  \begin{lstlisting}
    module analyzer
    using Plots
    using Printf
    using StatsBase
    using Statistics
    import PyPlot
    
    pyplot()
    default(show = true)
    
    function analyze(tags, spinning_flag; interval = 100e-6)
        machine_unit = 81e-12
        total_time = tags[length(tags)]
    
        interval_m_u = Int(floor(interval / machine_unit))
        group_num = Int(floor(total_time / interval_m_u))
        grouped = Array{Int}(undef, group_num + 1)
        tag_time = 0
        i = 1
        k = 1
    
        # counting photons in each interval
        while (tag_time<total_time)
            count = 0
            while (tags[i] <= tag_time)
                count += 1
                i += 1
            end
            tag_time += interval_m_u
            grouped[k] = count
            k += 1
        end
        
        # building the histogram
        bin_num = findmax(grouped)[1] + 1
        hist = Array{Int}(undef, bin_num)
        fill!(hist, 0)
        i = 1
        while (i<=length(grouped))
            hist[grouped[i] + 1] += 1
            i += 1
        end
    
        # normalizing the histogram
        prob = hist / group_num
    
        # computing moments
        accum = 0
        i = 1
        for i = 1:bin_num
            accum += (i-1) * prob[i]
        end
        mu = accum
    
        var = 0
        sk_acc = 0
        kr_acc = 0 
        for i = 1:bin_num
            var += (i-1 - mu)^2 * prob[i]
            sk_acc += (i-1 - mu)^3 * prob[i]
            kr_acc += (i-1 - mu)^4 * prob[i]
        end
        sigma = sqrt(var)
        sk = sk_acc / sigma^3
        kr = kr_acc / sigma^4
    
        first_n = bin_num
    
        # plotting
        fig = Plots.bar(1:first_n, prob[1:first_n], show=true,
         label = "empirical", size = (800, 300))
        
        # selecting theoretical statistic and drawing it 
        if spinning_flag == true
            Plots.scatter!(1:first_n,
             [bose_einst_pdf((x-1), mu) for x in 1:first_n],
             label = "Bose-Einstein pdf")
        else
            Plots.scatter!(1:first_n,
             [poisson_pdf((x-1), mu) for x in 1:first_n],
             label = "Poisson pdf")
        end
        display(fig)
    
        # saving png
        if spinning_flag == 1
            str = "spin"
        else
            str = "still"
        end
        savefig(fig, string("./images/", str,"-",
                Int(ceil(interval*1e6)), "us"))
        
        return [mu, var, sk, kr]
    end
    
    function poisson_pdf(x, mu)
        return (mu)^x/factorial(big(x)) * exp(-mu)
    end
    
    function bose_einst_pdf(x, mu)
        return 1/(mu + 1) * (mu/(mu+1))^x
    end
    
    function poisson_moments(mu)
        return [mu, mu, 1/sqrt(mu), (1+3mu)/mu]
    end
    
    function bose_ein_moments(mu)
        sigma = sqrt(mu + mu^2)
        return [mu,
                sigma^2,
                (mu + 3*mu^2 + 2*mu^3)/sigma^3,
                (mu + 10*mu^2 + 18*mu^3 + 9*mu^4)/sigma^4]
    end
    
    function run()
        PyPlot.clf()
        file_number = 10
        still = [string("./data/still/Part_", i, ".txt") 
                for i in 0:file_number-1]
        spin = [string("./data/spinning/Part_", i, ".txt") 
               for i in 0:file_number-1]

        still_data = Vector{Int}(undef, 1)
        spin_data = Vector{Int}(undef, 1)
        fill!(still_data, 0)
        fill!(spin_data, 0)
    
        # overflow values to be added to subsequent dataset tags
        of_value_1 = 0
        of_value_2 = 0
    
        for i = 1:file_number
            a_1 = readlines(still[i])
            a_2 = readlines(spin[i])
            b_1 = [split(x, ",")[1] for x in a_1]
            b_2 = [split(x, ",")[1] for x in a_2]
    
            # drop the first hit with time zero to preserve
             the exponential interarrival time
            tags_1 = [parse(Int, x) for x in b_1][2:length(b_1)]
                     .+ of_value_1
            tags_2 = [parse(Int, x) for x in b_2][2:length(b_2)]
                     .+ of_value_2
    
            of_value_1 = tags_1[length(tags_1)]
            of_value_2 = tags_2[length(tags_2)]
    
            append!(still_data, tags_1)
            append!(spin_data, tags_2)
        end
    
        println(still_data[1:10])
        println("Total acquisition time for still: ",
                 still_data[length(still_data)] * 81e-12)
        println("Total acquisition time for spinning: ",
                 spin_data[length(spin_data)] * 81e-12)
    
        range = [15, 100] .* 1e-6
        for interv in range
            println("--> interval time = ", interv)
            
            exp_still = analyze(still_data, 0; interval = interv)
            theo_still = poisson_moments(exp_still[1])
            println("Empirical still   :", exp_still)
            println("Theoretical still :", theo_still)
            println("Errors            :", 
                    (exp_still - theo_still)./theo_still, "\n")
    
            exp_spin = analyze(spin_data, 1; interval =  interv)
            theo_spin = bose_ein_moments(exp_spin[1])
            println("Empirical spin.   :", exp_spin)
            println("Theoretical spin. :", theo_spin)
            println("Errors            :",
                    (exp_spin - theo_spin)./theo_spin, "\n")
        end
    
    end
    end
  \end{lstlisting}
\end{mdframed}

\end{document}